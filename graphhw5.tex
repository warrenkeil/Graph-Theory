\documentclass{article}
\usepackage{pgf,tikz,tikzscale} 
\usepackage{amssymb}
\usepackage{tcolorbox}
\usepackage{xcolor}
\usepackage[utf8]{inputenc}
\usepackage[english]{babel}
\usepackage{multicol}
\usepackage{enumerate}	
\usepackage{graphicx,lipsum,pgfplots} 
\usepackage{amsmath, amsthm}                 
\usepackage[top=1in,bottom=1in, left=1in, right=1in] {geometry}  
\usepackage{fancyhdr}       
\usepackage{blkarray}


\pagestyle{fancy}              
\lhead{Math 5563 \newline Graph Theory HW5, Ch4}   
\rhead{Warren Keil}







\begin{document}
\setlength{\parindent}{0cm}   %%%%%%%% KEEP THIS  for block style paragraphs. 



%\textbf{11a.}  Suppose \(G\) is a graph on \(n\leq5\) vertices such that \(G\) is not coplanar. Prove that \(G=K_5\) or \(G=K_5^c\). 


%\vspace{3mm}
%\textit{Proof.} Let \(G\) be a graph on \(n\leq5\) vertices such that \(G\) is not coplanar. 

\textbf{11b.} Exhibit a coplanar graph \(G\) on 6 vertices such that both \(G\) and its complement are connected. 

\vspace{3mm}
\textit{Solution.} Observe the graph of \(G\) and its complement below. 


\vspace{60mm}

\textbf{14a.} Let \(G=P_3 \vee C_4\). Prove that \(G\) is not planar. 

\vspace{3mm}
\textit{Proof.} Let \(G=P_3 \vee C_4\). Recall Theorem 4.19 that states, 
\[
\gamma(G) \geq \frac m6 - \frac n2 + 1
\]
where \(\gamma(G)\) is the genus of \(G\), the minimum number of overpasses required to embed \(G\) into a surface. We notice that \(G\) has 7 vertices, and \(P_3\) and \(C_4\) have a combined 6 edges before we join them. After we join these graphs to construct \(G\), we have to add 12 edges to end up with 18 edges. Thus, 
\begin{align*}
\gamma(G) &\geq \frac{18}{6} - {7}{2} +1 \\
&=3- 3.5 + 1\\
&= .5
\end{align*}
And since \(\gamma(G) \geq .5\), then we know \(G\) cannot be planar since the genus of any planar graph is zero. 
\begin{flushright}
\(\qed\)
\end{flushright}







\end{document}
